\documentclass[10pt]{beamer}
\usepackage[utf8]{inputenc}
\usepackage[T1]{fontenc}
\usepackage{lmodern}
\usepackage{microtype}
\usepackage[]{amsmath}
\usepackage{amssymb}
\usepackage{amsfonts}
\usepackage{appendixnumberbeamer}
\usepackage[czech]{babel}
\usepackage[]{hyperref}
\usepackage{booktabs}
\usepackage{siunitx}
\usepackage[]{graphicx}
\usepackage[figurename=]{caption}
\usepackage[
    backend=biber
    ,style=iso-numeric
    ,autolang=other
    ,pagetotal=true
    ,sortlocale=cs_CZ
    ,bibencoding=UTF8
    ,spacecolon=false
    ,block=space
]{biblatex}
\addbibresource{citace.bib}

\usetheme[sectionpage=none, numbering=fraction, progressbar= head, block=fill]{metropolis}
\author{Michal Šesták}
\title{Multi-kompártmentový přístup ke kvantifikaci  objemové rychlosti přísunu zdrojů radonu do budov s využitím měřené intenzity větrání pomocí techniky indikačních plynů}
\subtitle{}
\institute{}
%\institute{Fakulta jaderná a fyzikálně inženýrská, ČVUT v Praze\\[0.5em]
%Vedoucí práce: Ing. Iva Ambrožová, Ph. D.}
\date{6. května 2019}
\begin{document}

\maketitle

%\begin{frame}{Obsah}
    %\tableofcontents
%\end{frame}
\begin{frame}{Osnova}
\item Úvod
\item Radon (Co to je, proc je nebezpecny, jeho dcery, co ovlivnuje jeho prisun do bytu, jak se meri, obrana proti nemu)
\item Rovnice
\item Model
\item Měření průtoků vzduchů
\end{frame}

\begin{frame}{Rovnice}
    \small
    \begin{equation}
        \dot{a_i}=\frac{1}{V_i}\left( \sum^n_{j=1}a_j k_{ji}-\sum^n_{j=1}a_i k_{ij}-(\lambda+k_i)a_i+Q_i \right)
        \label{eq:odvozovani}
    \end{equation}
        %\caption{<+Caption text+>}
        %\label{tab:<+label+>}
    \begin{table}
        \centering
        \begin{tabular}{lll}
            $a_i$ & koncentrace radonu v $i$-té zóně& [\si{Bq/m^3}] \\
            $V_i$ & objem $i$-té zóny& [\si{m^3}] \\
            $k_{ij}$ & objemový průtok vzduchu z $i$-té zóny do $j$-té zóny& [\si{m^3/hod}]\\
            $\lambda$ & přeměnová konstanta radonu& [\si{1/hod}]\\
            $k_i$ & výměna vzduchu $i$-té zóny& [\si{1/hod}] \\
            $Q_i$ & přísun radonu do $i$-té zóny& [\si{Bq/hod}] \\
        \end{tabular}
    \end{table}
\end{frame}

\begin{frame}{Rovnice}
    \small
    \begin{align}
        V_i\dot{a_i}&=\sum^n_{j=1}a_j k_{ji}-\sum^n_{j=1}a_i k_{ij}-(\lambda+k_i)a_i+Q_i\quad&\text{první varianta}\\
        V_i\dot{a_i}&=\sum^{n+1}_{j=1}a_j k_{ji}-\sum^{n+1}_{j=1}a_i k_{ij}-\lambda a_i+Q_i\quad&\text{druhá varianta}
        \label{eq:rovnice}
    \end{align}
    \begin{itemize}
        \item druhá varianta v případě blízkosti uranových hald atd.
    \end{itemize}
\end{frame}


\begin{frame}{Měření průtoků vzduchu mezi zónami}
    \small
    \begin{itemize}
        \item indikační plyny = perfluorokarbony
            \begin{itemize}
                \item netoxické, inertní, čisté, bezbarvé, nehořlavé a neradioaktivní plyny. 
                \item v přírodě se nevyskytují
            \end{itemize}
        \item vyvíječe
        \item integrální detektory 
    \end{itemize}
\end{frame}

\begin{frame}{Modelový příklad 1}
        \centering
        \includegraphics[width=.9\textwidth]{zony.png}
        \cite{japonci}       
\end{frame}

\begin{frame}{Modelový příklad 2}
        \centering
        \includegraphics[width=.9\textwidth]{zony2.png}
        \cite{japonci2}       
\end{frame}

% různá nastavení
%\metroset{sectionpage=simple,none}
%\begin{overlayarea}{\textwidth}{\textheight}\end{overlayarea}


\begin{frame}{Reference}
    \nocite{*}
\renewcommand*{\bibfont}{\tiny}
\printbibliography
\end{frame}
\end{document}
